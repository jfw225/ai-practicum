\documentclass[10pt]{article}

\usepackage{tex-assets/sbc-template} 
\usepackage{graphicx,url}
\usepackage{url}
\usepackage[utf8]{inputenc} 
\usepackage[T1]{fontenc}
\usepackage[normalem]{ulem}
\usepackage[hidelinks]{hyperref}

\usepackage{natbib}
\usepackage{amssymb} 
\usepackage{mathalfa} 
\usepackage{algorithm} 
\usepackage{algpseudocode} 
\usepackage[table]{xcolor}
\usepackage{array}
\usepackage{titlesec}
\usepackage{mdframed}
\usepackage{listings}

\usepackage{amsmath} 
\usepackage{booktabs}

\usepackage{indentfirst}
\usepackage{wrapfig}

\urlstyle{same}

\newcolumntype{L}[1]{>{\raggedright\let\newline\\\arraybackslash\hspace{0pt}}m{#1}}
\newcolumntype{C}[1]{>{\centering\let\newline\\\arraybackslash\hspace{0pt}}m{#1}}
\newcolumntype{R}[1]{>{\raggedleft\let\newline\\\arraybackslash\hspace{0pt}}m{#1}}

\newcommand\Tstrut{\rule{0pt}{2.6ex}} 
\newcommand\Bstrut{\rule[-0.9ex]{0pt}{0pt}} 
\newcommand{\scell}[2][c]{\begin{tabular}[#1]{@{}c@{}}#2\end{tabular}}

\usepackage[nolist,nohyperlinks]{acronym}

\newcommand{\baseline}[0]{\textit{\textbf{baseline design}}}
\newcommand{\alternative}[0]{\textit{\textbf{alternative design}}}

\title{Diagnosing Alzheimer's Disease from Brain fMRIs using Deep Learning}

\author{Jacob Diaz (jld349@cornell.edu), Joseph Whelan (jfw225@cornell.edu)}

\address{CS 4701, Practicum in Artificial Intelligence, Fall 2022, Cornell University}



\begin{document} 
	
	\maketitle
	
	\section{Introduction}
	The current standard for diagnosing Alzheimer's Disease requires significant time and resources. According to \cite[Mayo Clinic]{alz_desc}, doctors perform a physical exam, neurological exam, and cognitive tests to 
	"judge functional abilities and identify behavioral changes." Included in this process is magnetic resonance imaging (MRI) of the brain which produces a detailed view of the brain's structure in the form of 2d and/or 3d scan. In some instances functional MRIs (fMRI) are used is used to capture 3d images of oxygen levels in the brain over time. Currently there is research that investigates the use of covolutional neural networks on MRI scans and frames of fMRI scans. In particular these models are good at capturing the spatial dimention of the data to classify Alzheimer's in patients. Additionally, recurrent neural networks (RNN) are used to look at fMRI scans over time for fixed number of points in the brain. Both CNN and RNN are effective approaches for this classification problem, but they fail to look at the spatial and temporal aspects of fMRI data holistically. We seek to train a model that uses a hybrid of CNN and RNN to simeltaneously account for both the spatial and temporal aspects of fMRI data to diagnose Alzheimer's Disease. Using our model, we will build a web application that allows users to upload fMRI scans and receive an output classification which they can use to make a better informed diagnosis.
	

	\section{Approach}
	The first major hurdle with any machine learning problem is finding a suitable dataset. Fortunately, we found a dataset with approximately 5,000 images of brain MRIs with annotations indicating the severity of a patient's Alzheimer's Disease \citep{kaggle_dset}. We have also applied for access to some datasets that contain 3d MRI and fMRI scans of Alzheimer's patients. 

	Given that we already have an annotated dataset, the next step is analyzing the data and building an appropriate model architecture. We currently plan on using a hybrid model, such as ConvLSTM, to perform our classification. However, we are still exploring other hybrid model architectures. We will evaluate our model based on its classification accuracy
	($\frac{TP + TN}{TP + FN + TN + FP}$)
	as well as AUC-ROC. We choose these two metrics because they will help us  compare our results to other publications.	

	Once we have a working model, we will build a web application using TypeScript and React which will allow medical professionals to upload fMRI scans and receive a diagnosis. Upon upload, these images will be sent to a Python backend which will perform inference using the model. The backend will then return the classification to the frontend which will display it to the user.

	\pagebreak
	\section{Timeline}
	We will be meeting at least once a week until the final presentation to discuss our progress, plan for the next week, and work on any upcoming deadlines. Our tentative timeline is as follows:
	\begin{table}[!ht]
		\centering
		\begin{tabular}{|| c | p{80mm} ||} 
			\hline
			Date & Goal \\ [0.5ex] 
			\hline\hline
			Sep 30 & Turn in project proposal. \\ [0.5ex]
			\hline
			Oct 7 & Finish literature review and decide on the best hybrid model. \\ [0.5ex]
			\hline
			Oct 12 & Begin setup for training and testing. \\ [0.5ex]
			\hline
			Oct 21 & Have working model (not necessarily best one) that performs reasonably well relative to other publications. \\ [0.5ex]
			\hline
			Nov 1 & Start to optimize model to boost performance. Build the web frontend using TypeScript and React.\\ [0.5ex]
			\hline
			Nov 7 & Build the web backend using Python. This will include the model inference and the API. \\ [0.5ex]
			\hline
			Nov 14 & Start working on the final presentation powerpoint. \\ [0.5ex]
			\hline
			Nov 22 & Finish final presentation. \\ [0.5ex]
			\hline
		\end{tabular}
		\caption{\label{tab:timeline} Timeline for the Completion of our Project}
	\end{table}

    \bibliographystyle{apalike}
	\bibliography{tex-assets/references}
	
\end{document}
